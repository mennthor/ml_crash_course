\documentclass[9pt, table]{beamer}
  \usetheme[block=fill]{metropolis}  % https://github.com/matze/mtheme

\usepackage{amsmath}
\usepackage{amssymb}
\usepackage{mathtools}
\usepackage{xcolor}
\usepackage{csquotes}
\usepackage{url}

\title{Machine Learning}
\date{}
\author{Thorben Menne}

\newcommand{\deriv}[1]{\mathrm{d}#1}
\newcommand{\dderiv}[2]{\frac{\mathrm{d}#1}{\mathrm{d}#2}}
\newcommand{\del}[1]{\partial #1}
\newcommand{\ddel}[2]{\frac{\partial #1}{\partial #2}}

\newcommand{\tcw}[1]{\textcolor{white}{#1}}
\newcommand{\tcb}[1]{\textcolor{black}{#1}}
\newcommand{\ccw}{\cellcolor{white}}
\newcommand{\ccga}{\cellcolor{black!50}}
\newcommand{\ccgb}{\cellcolor{black!40}}
\newcommand{\ccgc}{\cellcolor{black!30}}
\newcommand{\ccb}{\cellcolor{black}}

% Buffer slides for subsections (tex.stackexchange.com/questions/117658)
\AtBeginSubsection[]{
  \begin{frame}
    \subsectionpage
  \end{frame}
}


% Main title, author etc. in header
\subtitle{Part I -- Math and Minimisers}

\begin{document}
  \maketitle

% %%%%%%%%%%%%%%%%%%%%%%%%%%%%%%%%%%%%%%%%%%%%%%%%%%%%%%%%%%%%%%%%%%%%%%%%%%%%%
% MATH
% %%%%%%%%%%%%%%%%%%%%%%%%%%%%%%%%%%%%%%%%%%%%%%%%%%%%%%%%%%%%%%%%%%%%%%%%%%%%%
  \section{Math}

% vectors and matrices %%%%%%%%%%%%%%%%%%%%%%%%%%%%%%%%%%%%%%%%%%%%%%%%%%%%%%%%
  \subsection{Vectors and Matrices}

% <Vectors>
  \begin{frame}{Vectors}
    \begin{itemize}
      \item Vectors have both magnitude ($|\vec{v}| \geq 0$) and direction
        \rightarrow \enquote{arrow}
      \item Vectors are represented by components indicating their length
        along the basis directions
      \item Usually we write column vectors.
        \enquote{rotate} vector up to get the transposed version
        \begin{equation*}
          \left(v_1, v_2, \dots, v_n\right)^\mathrm{T} \coloneqq
          \begin{pmatrix} v_1 \\ v_2 \\ \vdots \\ v_n \end{pmatrix}
        \end{equation*}
      \item Vectors can be added by putting the end of one vector at the tip
        of another one. Equivalent to summing each individual components
        \begin{equation*}
          \vec{v} + \vec{w} =
            \left(v_1 + w_1, v_2 + w_2, \dots, v_n + w_n\right)^\mathrm{T}
        \end{equation*}
    \end{itemize}
  \end{frame}

  \begin{frame}{Vectors}
    \begin{itemize}
      \item Scalar multiplication:
        Multiply every component with the scalar $\alpha$
        \begin{equation*}
          \alpha \vec{v} =
            \alpha \left(v_1, v_2, \dots, v_n\right)^\mathrm{T} =
            \left(\alpha v_1, \alpha v_2, \dots, \alpha v_n\right)^\mathrm{T}
        \end{equation*}
      \item Norm (length) of a vector
        \begin{equation*}
          |\vec{v}| = \sqrt{\vec{v}\cdot \vec{v}} =
            \sqrt{v_1^2 + v_2^2 + \dots + v_n^2}
        \end{equation*}
      \item Scalar product (also \enquote{inner product}):
        Multiply each element and sum up, produces a single number $\beta$
        \begin{equation*}
          \vec{v}\cdot \vec{w} =
            v_1 w_1 + v_2 w_2 + \dots + v_n w_n =
            \sum_{i=1}^{N} v_i w_i = \beta
        \end{equation*}
        Alternatively the following form is used to get the angle $\phi$
          between two vectors
        \begin{equation*}
          \vec{v}\cdot \vec{w} \coloneqq |\vec{v}||\vec{w}|\cos(\phi)
        \end{equation*}

    \end{itemize}
  \end{frame}
% </Vectors>

% <Vector Exercise>
  \begin{frame}{Vectors}
    \begin{exampleblock}{Exercise}
      We have two 4D vectors
      \begin{equation*}
        \vec{v} = \left(1, -2, 5, 0\right)^\mathrm{T}
        \quad\mathrm{and}\quad
        \vec{w} = \left(2, 2, 3, -1\right)^\mathrm{T}
      \end{equation*}
      \begin{enumerate}
        \item Calculate the lengths $|\vec{v}|$, $|\vec{w}|$
        \item Calculate the sum $\vec{v} + \vec{w}$ and difference $\vec{v} - \vec{w}$
        \item Calculate the scalar product $\vec{v}\cdot \vec{w}$
        \item What is angle $\phi$ in degrees between both vectors?
        \item Find a vector $\vec{u}\neq \vec{0}$ which is orthogonal to $\vec{v}$
      \end{enumerate}
      The \texttt{numpy} package in Python has built-in support for vector
      calculations.
      Use \texttt{numpy.dot}, \texttt{numpy.linalg.norm},
      \texttt{numpy.rad2deg}, etc. to check your calculations above or code the
      functionality yourself.
      See \url{https://numpy.org/devdocs/reference/index.html} for help.
    \end{exampleblock}
  \end{frame}
% </Vector Exercise>

% <Matrices>
  % Intro, vector identification
  \begin{frame}{Matrices}
    \begin{itemize}
      \item Matrices are 2D collections of numbers
      \item A $(n, m)$ matrix $\underline{M}$ is defined as a tuple of numbers
        $a_{ij}$ ($i=1,\dots,m$, $j=1,\dots,n$)
        \begin{equation*}
          \underline{M} =
          \begin{pmatrix}
            a_{11} & a_{12} & \dots & a_{1n} \\
            a_{21} & a_{22} & \dots & a_{2n} \\
            \vdots & \vdots & & \vdots \\
            a_{m1} & a_{m2} & \dots & a_{mn} \\
          \end{pmatrix}
        \end{equation*}
      \item We can identify column vectors with $(m, 1)$ matrices and row
        vectors with $(1, n)$ matrices
        \begin{gather*}
          \underline{M}_{1,n} = \left(v_1, \dots, v_n\right)
            \leftrightarrow \vec{v}^\mathrm{T} \\
          \underline{M}_{m, 1} =
            \begin{pmatrix}w_1 \\ \vdots \\ w_m \end{pmatrix}
            \leftrightarrow \vec{w}
        \end{gather*}
    \end{itemize}
  \end{frame}

  % Scalar multiplication, matrix product
  \begin{frame}{Matrices}
    \begin{itemize}
      \item Multiplication with a scalar $\alpha$ is done by multiplication of
        every element with that scalar
        \begin{equation*}
          \alpha \underline{M} =
          \begin{pmatrix}
            \alpha a_{11} & \alpha a_{12} & \dots & \alpha a_{1n} \\
            \alpha a_{21} & \alpha a_{22} & \dots & \alpha a_{2n} \\
            \vdots & \vdots & & \vdots \\
            \alpha a_{m1} & \alpha a_{m2} & \dots & \alpha a_{mn} \\
          \end{pmatrix}
        \end{equation*}
      \item Matrix products $\underline{C} = \underline{A}\,\underline{B}$ are
        defined component-wise by
        \begin{equation*}
          c_{ij} = \sum_{k=1}^{n} a_{ik}b_{kj}
        \end{equation*}
        where $\underline{A}$ is a $(m, n)$, $\underline{B}$ a $(n, p)$ and
        $\underline{C}$ a $(m, p)$ matrix
        \begin{itemize}
          \item This is usually called
            \enquote{multiply row $i$ of $\underline{A}$ by
                     column $j$ of $\underline{B}$}
          % \item The inner index range must match, we can multiply any 2 matrices
          %   where the number of rows of the first on equals the number of columns
          %   in the second one
        \end{itemize}
      \begin{exampleblock}{Example: Matrix times vector}
        \begin{equation*}
          \begin{pmatrix}
            a_{11} & a_{12} \\ a_{21} & a_{22} \\ a_{31} & a_{32}
          \end{pmatrix}_{(3,2)}
            \begin{pmatrix}v_1 \\ v_2\end{pmatrix}_{(2,1)} =
          \begin{pmatrix}
            a_{11}v_1 + a_{12}v_2 \\
            a_{21}v_1 + a_{22}v_2 \\
            a_{31}v_1 + a_{32}v_2
          \end{pmatrix}_{(3,1)}
        \end{equation*}
      \end{exampleblock}
    \end{itemize}
  \end{frame}

  % Transpose, inverse
  \begin{frame}{Matrices}
    \begin{itemize}
      \item The transpose of a matrix is obtained by switching
        elements $a_{ij}\leftrightarrow a_{ji}$
        \begin{equation*}
          \underline{M}^\mathrm{T} =
          \begin{pmatrix}
            a_{11} & a_{12} & \dots & a_{1n} \\
            a_{21} & a_{22} & \dots & a_{2n} \\
            \vdots & \vdots &  & \vdots \\
            a_{m1} & a_{m2} & \dots & a_{mn} \\
          \end{pmatrix}^\mathrm{T} =
          \begin{pmatrix}
            a_{11} & a_{21} & \dots & a_{m1} \\
            a_{12} & a_{22} & \dots & a_{m2} \\
            \vdots & \vdots &  & \vdots \\
            a_{1n} & a_{2n} & \dots & a_{mn} \\
          \end{pmatrix}
        \end{equation*}
      \item The inverse $\underline{M}^{-1}$ of a matrix is defined so that
        \begin{equation*}
          \underline{M}^{-1}\underline{M} =
          \underline{M}\,\underline{M}^{-1} = \underline{1}
        \end{equation*}
        where $\underline{1}$ is the identity matrix
        ($1$ on diagonal, $0$ everywhere else)
        \begin{itemize}
          \item The inverse can be found by solving a set of linear equations
            using the definition above
          \item Note: Not every matrix has an inverse!
          \item For $(2,2)$ matrices the inverse can be explicitly calculated by
            \begin{equation}
              \begin{pmatrix} a & b \\ c & d \end{pmatrix}^{-1} =
              \frac{1}{ad-bc}
                \begin{pmatrix} d & -b \\ -c & a \end{pmatrix}
            \end{equation}
        \end{itemize}
    \end{itemize}
  \end{frame}
% </Matrices>

% <Matrix Exercises>
  \begin{frame}{Matrices}
    \begin{exampleblock}{Exercise}
      \begin{enumerate}
        \item We have 2 matrices $\underline{A}, \underline{B}$
          and a vector $\vec{v}$
          \begin{equation*}
            \underline{A} =
              \begin{pmatrix} 1 & 2 & 3 \\ 3 & 2 & 1 \end{pmatrix}
            \,\mathrm{, }\quad
            \underline{B} =
              \begin{pmatrix} 0 & 2 \\ 1 & -1 \\ 0 & 1 \end{pmatrix}
            \,\mathrm{, }\quad
            \vec{v} = \begin{pmatrix} 4 \\ 5 \\ 6 \end{pmatrix}
          \end{equation*}
          Calculate $\underline{A}\,\underline{B}$,
          $\underline{B}\,\underline{A}$,
          $\underline{A}\,\vec{v}$ and $\vec{v}^\mathrm{T}\,\underline{B}$.
        \item Calculate the inverses (if existing) of
          \begin{equation*}
            \underline{A} =
              \begin{pmatrix} 1 & 2 \\ 4 & 2 \end{pmatrix}
            \quad\mathrm{, }\quad
            \underline{B} =
              \begin{pmatrix} 1 & 2 \\ 2 & 4 \end{pmatrix}
          \end{equation*}
        \item We have a matrix $\underline{M} = \underline{1}$
          and a vector $\vec{v}^\mathrm{T} = \left(v_1, v_2, v_3\right)$.
          Calculate
          \begin{equation*}
            \vec{v}^\mathrm{T}\, \underline{M}\, \vec{v}
          \end{equation*}
      \end{enumerate}
    Again, use \texttt{numpy} or own code to check your results.
    \end{exampleblock}
  \end{frame}
% </Matrix Exercises>

% <Convolution>
  \begin{frame}{Convolution}
    \begin{itemize}
      \item Mathematically
        \begin{equation*}
          \left(f * g\right)(t) \coloneqq
            \int_{-\infty}^\infty f(\tau)g(t-\tau)\deriv{\tau}
        \end{equation*}
      \begin{itemize}
        \item Intuitive: $f$ is a \enquote{signal}, $g$ is a \enquote{response}.
          At time $t$ the convolution gives the (causal, thus we mirror $g$)
          system response for input $f$ and response $g$ by summing (integrating)
          all infinitesimal mini-responses up to $t$
          \item Imagine $f$ being decomposed in \enquote{delta} peaks and $g$
            acts independently on each input impulse
      \end{itemize}
      \item Here we need the discrete version, replacing integrals with sums
      and functions with arrays of numbers ($f(x)\rightarrow f[i]$)
      \begin{equation*}
        (f*g)[i] = \sum_{k=-\infty}^{\infty} f[k]g[i-k]
      \end{equation*}
      \item Note: In real application the kernel or the signal has some finite support
      (there are no infinite length arrays), so the sum will be constrained by
      some range $\{-N, -N+1, \dots, N-1, N\}$.
      \begin{itemize}
        \item Finite support leads to problems at the borders of $f$
        \item Multiple strategies exist, eg. padding with zero,
          mirroring at the border, cyclic repetition or
          repeating the outermost element
      \end{itemize}
    \end{itemize}
  \end{frame}
% </Convolution>

% <Convolution>
  \begin{frame}{Convolution}
    \begin{itemize}
      \item In two dimensions, convolutions work the same way
        \begin{equation*}
          (f*g)[i, j] =
          \sum_{k=-\infty}^{\infty} \sum_{l=-\infty}^{\infty} f[k,l]g[i-k, j-l]
        \end{equation*}
      \item Same here with the finite support
      \item In image applications, the response is often called kernel or filter
      \item Don't forget to flip the kernel horizontally and vertically before
        computing the elementwise multiplication
      \item This is what is used in convolutional neural network layers (although
        technically they don't flip the kernel, but the weights are learned anyway)
    \end{itemize}
  \end{frame}
% </Convolution>

% <Convolution Exercises>
  \begin{frame}{Convolution}
    \begin{exampleblock}{Exercise}
    For the following exercises, pad the signal function with zeroes where
    appropriate. Additionally you can also try cyclic, mirrored or \enquote{repeated}
    edges or only calculate the convolution where the signal is valid.
      \begin{enumerate}
        \item Consider a discrete signal
          \begin{equation*}
            f[i] = 3\delta[i]
          \end{equation*}
          (value of $3$ at position $i=0$) and a response
          \begin{equation*}
            g[i] = \left\{g[i=0] = 2, g[i=1] = 1\right\}
          \end{equation*}
          Compute the convolution $f*g$ for all non-zero components.
        \item Now we have a more complicated signal
          \begin{equation*}
            f[i] = \left\{f[i=0] = 3, f[i=1] = 4, f[i=2] = 5\right\}
          \end{equation*}
          Compute $f*g$ with the same kernel $g$ as above. \\
          Tip: \emph{For grasping the concept of why the kernel is flipped, try
          to imagine how each signal contribution triggers the response. The
          response needs some time to arrive at the current index which in the
          end gives you the flipped kernel in the formula.}
      \end{enumerate}
    \end{exampleblock}
  \end{frame}
% </Convolution Exercises>

% <Convolution Exercises>
  \begin{frame}{Convolution}
    \begin{exampleblock}{Exercise}
      \begin{enumerate}
        \item[3] Now we try a 2D convolution.
          Convolve the following $8x8$ "grayscale image"
          \begin{equation*}
            \mathrm{Img}_1 = \quad \begin{matrix}
                \ccb \tcw{0.0} & \ccga 0.5 & \ccga 0.5 & \ccga 0.5 & \ccga 0.5 & \ccga 0.5 & \ccga 0.5 & \ccb \tcw{0.0} \\
                \ccb \tcw{0.0} & \ccb \tcw{0.0} & \ccga 0.5 & \ccga 0.5 & \ccga 0.5 & \ccga 0.5 & \ccb \tcw{0.0} & \ccb \tcw{0.0} \\
                \ccb \tcw{0.0} & \ccb \tcw{0.0} & \ccb \tcw{0.0} & \ccb \tcw{0.0} & \ccb \tcw{0.0} & \ccb \tcw{0.0} & \ccb \tcw{0.0} & \ccb \tcw{0.0} \\
                \ccb \tcw{0.0} & \ccb \tcw{0.0} & \ccb \tcw{0.0} & \ccb \tcw{0.0} & \ccb \tcw{0.0} & \ccb \tcw{0.0} & \ccb \tcw{0.0} & \ccb \tcw{0.0} \\
                \ccb \tcw{0.0} & 1.0 & 1.0 & \ccb \tcw{0.0} & \ccb \tcw{0.0} & 1.0 & 1.0 & \ccb \tcw{0.0} \\
                \ccb \tcw{0.0} & \ccb \tcw{0.0} & \ccb \tcw{0.0} & \ccb \tcw{0.0} & \ccb \tcw{0.0} & \ccb \tcw{0.0} & \ccb \tcw{0.0} & \ccb \tcw{0.0} \\
                \ccb \tcw{0.0} & \ccb \tcw{0.0} & \ccgc 0.7 & \ccgb 0.6 & \ccgb 0.6 & \ccgc 0.7 & \ccb \tcw{0.0} & \ccb \tcw{0.0} \\
                \ccga 0.5 & \ccb \tcw{0.0} & \ccgc 0.7 & \ccgc 0.7 & \ccgc 0.7 & \ccgc 0.7 & \ccb \tcw{0.0} & \ccga 0.5
            \end{matrix}
          \end{equation*}
          with a $3x3$ top to bottom Sobel kernel (gradient kernel for edge finding):
          \begin{equation*}
            \begin{pmatrix}
              -1 & -2 & -1 \\
              0 & 0 & 0 \\
              1 & 2 & 1
            \end{pmatrix}
          \end{equation*}
        Interpret the results.
      \end{enumerate}
    \end{exampleblock}
  \end{frame}
% </Convolution Exercises>

% functions %%%%%%%%%%%%%%%%%%%%%%%%%%%%%%%%%%%%%%%%%%%%%%%%%%%%%%%%%%%%%%%%%%%
\subsection{Functions}

% <Functions>
  \begin{frame}{Functions}
    Hello, world!
  \end{frame}
% </Functions>

% <Function Composition>
  \begin{frame}{Function Composition}
    Hello, world!
  \end{frame}
% </Function Composition>

% <Function Composition Exercises>
  \begin{frame}{Function Composition Exercises}
    Hello, world!
  \end{frame}
% </Function Composition Exercises>

% <Derivatives>
  \begin{frame}{Derivatives}
    Hello, world!
  \end{frame}
% </Derivatives>

% <Derivatives Exercises>
  \begin{frame}{Derivatives Exercises}
    Hello, world!
  \end{frame}
% </Derivatives Exercises>

% <Chain Rule>
  \begin{frame}{Chain Rule}
    Hello, world!
  \end{frame}
% </Chain Rule>

% <Chain Rule Exercises>
  \begin{frame}{Chain Rule Exercises}
    Hello, world!
  \end{frame}
% </Chain Rule Exercises>

% <Taylor Expansion>
  \begin{frame}{Taylor Expansion}
    Hello, world!
  \end{frame}
% </Taylor Expansion>

% %%%%%%%%%%%%%%%%%%%%%%%%%%%%%%%%%%%%%%%%%%%%%%%%%%%%%%%%%%%%%%%%%%%%%%%%%%%%%
% MINIMIZERS
% %%%%%%%%%%%%%%%%%%%%%%%%%%%%%%%%%%%%%%%%%%%%%%%%%%%%%%%%%%%%%%%%%%%%%%%%%%%%%
  \section{Minimisation}
% <Analytical>
  \begin{frame}{Analytical}
    A
  \end{frame}
% </Analytical>

% <Bisection>
  \begin{frame}{Bisection}
    A
  \end{frame}
% </Bisection>

% <Newton Methods>
  \begin{frame}{Newton Methods}
    A
  \end{frame}
% </Newton Methods>

% <Gradient Descent>
  \begin{frame}{Gradient Descent}
    A
  \end{frame}
% </Gradient Descent>

% <Stochastic Gradient Descent>
  \begin{frame}{Stochastic Gradient Descent}
    A
  \end{frame}
% </Stochastic Gradient Descent>

% <Momentum Gradient Descent>
  \begin{frame}{Momentum Gradient Descent}
    A
  \end{frame}
% </Momentum Gradient Descent>

\end{document}